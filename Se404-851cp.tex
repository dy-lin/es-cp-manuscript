\documentclass[titlepage,11pt, oneside]{article}   	% use "amsart" instead of "article" for AMSLaTeX format
\usepackage{geometry}                		% See geometry.pdf to learn the layout options. There are lots.
\usepackage[utf8]{inputenc}
\geometry{letterpaper}                   		% ... or a4paper or a5paper or ... 
%\geometry{landscape}                		% Activate for rotated page geometry
%\usepackage[parfill]{parskip}    		% Activate to begin paragraphs with an empty line rather than an indent
\usepackage{graphicx}				% Use pdf, png, jpg, or eps§ with pdflatex; use eps in DVI mode
								% TeX will automatically convert eps --> pdf in pdflatex		
\usepackage{amssymb}
\usepackage{authblk}
\usepackage{indentfirst}
\usepackage{gensymb}
\usepackage{hyperref}
\usepackage[labelfont=bf]{caption}
%SetFonts
%\makeatletter
%\renewcommand{\maketitle}{\bgroup\setlength{\parindent}{0pt}
%\begin{flushleft}
%  \textbf{\@title}
%
%  \@author
%\end{flushleft}\egroup
%}
%\makeatother

\title{\textbf{Complete chloroplast genome sequence of \textit{Picea engelmannii}, isolate Se404-851 from western Canada\newline}}
\author{TBD}
\date{February 1, 2019}
\begin{document}
\maketitle
\begin{abstract}
\textit{Picea engelmannii}, is an Engelmann spruce tree grown in western Canada. Here we present the complete chloroplast genome sequence of the \textit{Picea engelmannii}, isolate Se404-851. This sequence contributes data to the study of the evolutionary phylogeny of the \textit{Picea} organisms, and consequently, facilitating the improvement of Canada’s forestry industry.
\end{abstract}

\section*{Genome Announcement}

As part of the \href{http://spruce-up.ca/en/}{SpruceUp} (1) and \href{https://www.smartforests.ca}{SMarTForests} (2) projects, we sequenced, assembled and annotated the chloroplast genome of \textit{Picea engelmannii} isolate Se404-851. This work contributes to improving the selection of spruce trees in breeding programs, a vital part of Canada’s forestry industry.
\newline
\par
The needle tissue sample was collected from Kalamalka Forestry Centre in British Columbia (36\degree17'60"N, 105\degree24'0"W; elevation: 298 m). We then sequenced the sample at the \href{http://www.bcgsc.ca}{British Columbia Genome Sciences Centre}.
\newline
\par
To sequence the sample, a modified version of TruSeq DNA PCR-Free kit (E6875-6877B-GSC, New England Biolabs) genome protocol was used to generate a 900-bp gap Illumina library on a Microlab NIMBUS liquid handling robot (Hamilton). Briefly, 5$\mu$g of genomic DNA was subjected to shearing by sonication (Covaris LE220) using a Duty Factor of 5 and Peak Incident Power of 450 for 70 seconds. The sonicated DNA products were concentrated with PCRClean DX magnetic beads (Aline Biosciences) and fractionated in 2 lanes of a 6\% PAGE gel to recover fragments greater than 700-bp for library preparation.  The isolated DNA fragments were end-repaired and bead-purified with a 1.8:1 ratio of beads, then A-tailed and ligated with full length indexed TruSeq adapters, and bead-purified. For quality check, an aliquot of the constructed library DNA was PCR amplified with Illumina universal primers to estimate the library gap size, using the Agilent 2100 Bioanalyzer HSDNA assay, while the library concentration was determined using KAPA qPCR Library Quantification kit (KK4824). The PCR-Free library was sequenced with paired-end 150 base reads on the Illumina HiSeqX platform using V4 chemistry according to manufacturer recommendations.
\newline
\par
For assembly, we subsampled the reads into a few million read pairs. ABySS v2.1.1 (3) assembled each subset. Then, BWA v0.7.17 (4) filtered for contigs greater than 500-bp that aligned with the reference chloroplast, \textit{Picea glauca} isolate PG29. In the 3M assembly, a resultant contig (\textit{n}=1; \textit{N\textsubscript{50}}=1743). QUAST v5.0.0 (5) revealed zero misassemblies and internal gaps.
\newline
\par
To be consistent, we modified our assembly to match the reference using BLAST v2.7.1 (6). We also introduced a gap, at the terminals of our assembly, which allowed circularization of the sequence upon running through the ABySS-sealer, therefore ensuring sequence completion. Finally, Pilon v1.22 (7) polished the final assembly.
\newline
\par
The Se404-851 chloroplast genome is 123,601-bp in length, with a GC content of 38.74\%. GeSeq (8) annotated 114 genes: 74 protein-coding, 36 tRNA-coding, 4 rRNA-coding genes, using all available \textit{Picea} chloroplast genomes as reference. We manually annotated four genes: \textit{rps12}, \textit{petB}, \textit{petD}, and \textit{rpl16}. OGDRAW v1.2 (9) generated the genome map. By introducing this new chloroplast genome, further analysis of phylogeny and evolution of these spruce trees can be conducted to achieve the project goals.
\newline
\newline
\textbf{Accession number(s).} The complete chloroplast genome sequence of \textit{Picea engelmannii}, isolate Se404-851 can be found in Genbank under \href{https://www.ncbi.nlm.nih.gov/nuccore/MK241981}{MK241981}. The tissue samples used can be found under BioSample: SAMN10388286; BioProject: PRJNA504036. The annotation NCBI references are as follows: \textit{Picea abies} (\href{https://www.ncbi.nlm.nih.gov/nuccore/NC_021456}{NC\_021456}), \textit{Picea asperata} (\href{https://www.ncbi.nlm.nih.gov/nuccore/NC_032367}{NC\_032367}), \textit{Picea glauca} isolate PG29 (\href{https://www.ncbi.nlm.nih.gov/nuccore/NC_028594}{NC\_028594}), \textit{Picea morrisonicola} (\href{https://www.ncbi.nlm.nih.gov/nuccore/NC_016069}{NC\_016069}), and \textit{Picea sitchensis} (\href{https://www.ncbi.nlm.nih.gov/nuccore/NC_011152}{NC\_011152}).

\section*{Acknowledgements}
This work was conducted as a part of the SpruceUp and SMarTForests projects.

\section*{References}
\begin{enumerate}
\setcounter{enumi}{2}
\item Jackman SD, Vandervalk BP, Mohamadi H, Chu J, Yeo S, Hammond SA, Jahesh G, Khan H, Coombe L, Warren RL, Birol I. 2017. ABySS 2.0: resource-efficient assembly of large 
genomes using a Bloom filter. Genome research 27:768-777.
\item Li H, Durbin R. 2009. Fast and accurate short read alignment with Burrows-Wheeler transform. Bioinformatics (Oxford, England) 25:1754-1760.
\item Gurevich A, Saveliev V, Vyahhi N, Tesler G. 2013. QUAST: quality assessment tool for genome assemblies. Bioinformatics (Oxford, England) 29:1072-1075.
\item Altschul SF, Gish W, Miller W, Myers EW, Lipman DJ. 1990. Basic local alignment search tool. Journal of Molecular Biology 215:403-410.
\item Walker BJ, Abeel T, Shea T, Priest M, Abouelliel A, Sakthikumar S, Cuomo CA, Zeng Q, Wortman J, Young SK, Earl AM. 2014. Pilon: an integrated tool for comprehensive microbial variant detection and genome assembly improvement. PloS one 9:e112963-e112963.
\item Tillich M, Lehwark P, Pellizzer T, Ulbricht-Jones ES, Fischer A, Bock R, Greiner S. 2017. GeSeq†- versatile and accurate annotation of organelle genomes. Nucleic acids research 45:W6-W11.
\item Lohse M, Drechsel O, Kahlau S, Bock R. 2013. OrganellarGenomeDRAW--a suite of tools for generating physical maps of plastid and mitochondrial genomes and visualizing expression data sets. Nucleic acids research 41:W575-W581.
\end{enumerate}

\end{document}